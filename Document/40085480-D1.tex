\documentclass[12pt]{extarticle}
%Some packages I commonly use.
\usepackage[english]{babel}
\usepackage{graphicx}
\usepackage{framed}
\usepackage[normalem]{ulem}
\usepackage{amsmath}
\usepackage{clrscode3e}
\usepackage{amsthm}
\usepackage{amssymb}
\usepackage{amsfonts}
\usepackage{enumerate}
\usepackage{indentfirst}
\usepackage[utf8]{inputenc}
\usepackage[top=1 in,bottom=1in, left=1 in, right=1 in]{geometry}
%A bunch of definitions that make my life easier
\newcommand{\matlab}{{\sc Matlab} }
\newcommand{\cvec}[1]{{\mathbf #1}}
\newcommand{\rvec}[1]{\vec{\mathbf #1}}
\newcommand{\ihat}{\hat{\textbf{\i}}}
\newcommand{\jhat}{\hat{\textbf{\j}}}
\newcommand{\khat}{\hat{\textbf{k}}}
\newcommand{\minor}{{\rm minor}}
\newcommand{\trace}{{\rm trace}}
\newcommand{\spn}{{\rm Span}}
\newcommand{\rem}{{\rm rem}}
\newcommand{\ran}{{\rm range}}
\newcommand{\range}{{\rm range}}
\newcommand{\mdiv}{{\rm div}}
\newcommand{\proj}{{\rm proj}}
\newcommand{\R}{\mathbb{R}}
\newcommand{\N}{\mathbb{N}}
\newcommand{\Q}{\mathbb{Q}}
\newcommand{\Z}{\mathbb{Z}}
\newcommand{\<}{\langle}
\renewcommand{\>}{\rangle}
\renewcommand{\emptyset}{\varnothing}
\newcommand{\attn}[1]{\textbf{#1}}
\theoremstyle{definition}
\newtheorem{theorem}{Theorem}
\newtheorem{corollary}{Corollary}
\newtheorem*{definition}{Definition}
\newtheorem*{example}{Example}
\newtheorem*{note}{Note}
\newtheorem{exercise}{Exercise}
\newcommand{\bproof}{\bigskip {\bf Proof. }}
\newcommand{\eproof}{\hfill\qedsymbol}
\newcommand{\Disp}{\displaystyle}
\newcommand{\qe}{\hfill\(\bigtriangledown\)}
\setlength{\columnseprule}{1 pt}
\title{F5:Gamma function}
\date{}
\author{Liangzhao Lin 40085480}
\begin{document}
\maketitle
\section{Introduction of Gamma function }
\setlength{\parindent}{2em}
The gamma function$^{[1]}$ $\Gamma \left( x \right)$ is a transcendental function, which is a kind of function that the factorial function on real numbers and expands on complex numbers.
\newline
\indent
In the field of real numbers, the domain of this function is $x$ can be any real number except 0 and negative integer. The co-domain of this function is $(-{\infty},+{\infty}).$The image of the gamma function is shown in Figure 1.
\begin{figure}[ht]

\centering
\includegraphics[scale=0.3]{gammafunction.png}
\caption{Gamma Function.  (Source: Google Images)}
\label{fig:label}
\end{figure}
\begin{enumerate}[(1)]

\item The gamma function is defined on the real number field as:$$\Gamma \left( x \right) = \int\limits_0^\infty {s^{x - 1} e^{ - s} ds}(x>0)$$
\item For a positive integer $x$, $\Gamma \left( x \right) = (x-1)!$
\item This function has recursive properties,  $\Gamma \left( x+1 \right) = x\Gamma\left( x \right) $
\end{enumerate}
\section{Requirements analysis}
\begin{enumerate}[\text{2.}1]
\item Functional requirement
\begin{enumerate}[\text{2.1.}1]
\item The user can enter any real number into the system.
\item The system can perform Gamma function calculation on the input $x$.
\item Display result. System shall display numbers to ten digits accuracy, the function shall display the result in scientific notation.
\item When the input entered by the user is 0 or a negative integer or other input outside the domain, the function shall return a result of ERROR.
\end{enumerate}
\item Non-Functional requirement
\begin{enumerate}[\text{2.2.}1]
\item Portability. As long as the input is real number, all kinds of calculators can use this function directly. And the function can be implemented and used in both windows and MacOs operating systems.
\item Performance.The calculation range is real number. Fast calculations, all inputs can produce results in less than a second.
\item Reliability. The function result is high precision and the result is accurate.
\end{enumerate}

\end{enumerate}

\section {Algorithm selection}
\indent
Values of the gamma function can be computed numerically with high precision using Stirling's approximation algorithm or the Lanczos approximation algorithm. 
\begin{enumerate}[\text{3.}1]
\item Stirling's approximation algorithm
\newline
In mathematics, Stirling's approximation$^{[2]}$ is an approximation for factorials. The Stirling formula used to calculate the gamma function is:$$\Gamma \left( x+1 \right)\approx \sqrt{2\pi x}(\frac{x}{e})^{x}$$
The Figure2 illustrate the comparison of Stirling's approximation with the factorial.
\begin{figure}[ht]
\centering
\includegraphics[scale=0.3]{figure2.png}
\caption{ \small Comparison of Stirling's approximation with the factorial.}
\end{figure}
\newline
\begin{enumerate}
\item[-]The advantage of Stirling's approximation algorithm is that when $x$ is large, their results are almost identical, moreover, even when $x$ is small, the result of the Stirling's approximation is also very accurate.
\newline
\item[-]The disadvantage is that this algorithm can not handle real numbers less than or equal to 0. 
\item[-]The pseudocode of Stirling's approximation algorithm is below:
\begin{codebox}
\Procname{$\proc{Stirling's approximation}(n)$}
\li $t$ = 1, $e$ = 2.718281828459, $pi$ = 3.14159265359
\li \For $i \gets 1$ \To $n$
\li     \Do
            $\id{t} \gets t*(n/e)$
\li         $i \gets i + 1$
        \End
\li $t = sqrt(2*pi*n)*t$
\li return $t$
\end{codebox}

\end{enumerate}
\item Lanczos approximation algorithm
\newline
The Lanczos approximation$^{[3]}$ is a method for computing the gamma function numerically,it is a practical alternative to the more popular Stirling's approximation for calculating the gamma function with fixed precision.
\begin{enumerate}

\item[-]The pseudocode of Lanczos approximation algorithm is below: 

\begin{codebox}
\Procname{$\proc{Lanczos approximation}(x)$}
\li $p=3.14159265359$
\li $t \gets (x-0.5)*log(x+4.5)-(4+4.5)$
\li $s \gets 1.0 + 76.18009173 / (x + 0) - 86.50532033 / (x + 1) + 24.01409822 / (x + 2)$ \\
$- 1.231739516 / (x + 3) + 0.00120858003 / (x + 4) - 0.00000536382 / (x + 5)$
\li $logGamma \gets t+log(s*sqrt(2*pi)$
\li $Gamma \gets exp(logGamma)$
\li return $Gamma$
\end{codebox}
\item[-]The advantage of Lanczos approximation algorithm is that this algorithm can calculate all the input in the gamma function domain.The precision of the calculation result can be adjusted and the result can be very accurate.
\item[-]The disadvantage is that the constant coefficient of each level in the formula is slightly more difficult to calculate, omitting the smallest coefficients does not result in a faster but slightly less accurate implementation.
\end{enumerate}
Overall, combined with the requirement analysis, Lanczos approximation algorithm is more suitable, it should be selected.
\end{enumerate}
\section{Reference}

\noindent[1] C. Lanczos, "A precision approximation of the gamma function," Journal of the Society for Industrial and Applied Mathematics, Series B: Numerical Analysis, vol. 1, (1), pp. 86-96, 1964.
\newline
[2] H. Robbins, "A remark on Stirling's formula," The American Mathematical Monthly, vol. 62, (1), pp. 26-29, 1955.
\newline
[3] P. J. Davis, "Leonhard Euler's integral: A historical profile of the Gamma function: In memoriam: Milton Abramowitz," The American Mathematical Monthly, vol. 66, (10), pp. 849-869, 1959. 
\end{document}
